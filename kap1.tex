\chapter{Einführung}

\section{Firefox OS}
Firefox OS ist ein mobiles Betriebssystem, welches komplett auf Web-Technologien aufsetzt\footnote{Mozilla Developer Network: Firefox OS Architektur. \url{https://developer.mozilla.org/de/Firefox_OS/Platform/Architektur} . Aufgerufen am 05. April 2015.}. Die Mozilla Foundation steht als Koordinator hinter dem Projekt. Unterstützt wird Mozilla von tausenden Freiwilligen weltweit. Das Ziel von Firefox OS ist es, eine Basis für kostengünstige Smartphones zur Verfügung zu stellen, welches zudem auf offenen Web Standards basiert. Webentwickler können ihre bestehenden Webseiten adaptieren und diese auf der Firefox OS Plattform ohne grösseren Anpassungen zur Verfügung stellen. Das erste Firefox OS Gerät wurde Anfang Juli 2013 durch Movistar in Spanien auf den Markt gebracht\footnote{Mozilla Blog: Mozilla and Partners Prepare to Launch First Firefox OS Smartphones! \url{https://blog.mozilla.org/blog/2013/07/01/mozilla-and-partners-prepare-to-launch-first-firefox-os-smartphones} . Aufgerufen am 05. April 2015.}. Bis Ende erstes Quartal 2016 werden Firefox OS Geräte in über 40 Länder verfügbar sein\footnote{Mozilla Blog: Firefox OS Proves Flexibility of Web: Ecosystem Expands with More Partners, Device Categories and Regions in 2015. \url{https://blog.mozilla.org/blog/2015/03/01/firefox-os-proves-flexibility-of-web-ecosystem}. Aufgerufen am 05. April 2015.}. Desweiteren wird Firefox OS nicht nur auf Smartphones, sondern auch auf Tablets und Fernseher verfügbar sein.

\section{Gaia}
Firefox OS besteht aus 3 Architekturschichten. Zuunterst befindet sich Gonk, der Kernel. Dieser startet Gecko, die Rendering Engine für Webseiten, welche auch in Firefox zum Einsatz kommt. Darauf aufbauend ist Gaia die Benutzeroberfläche von Firefox OS\footnote{Mozilla Developer Network: Introduction to Gaia. \url{https://developer.mozilla.org/en-US/Firefox_OS/Platform/Gaia/Introduction_to_Gaia}. Aufgerufen am 05. April 2015.}. Alles, was der Benutzer sieht, wird in Gaia umgesetzt. Gaia ist in HTML, CSS und JavaScript geschrieben. Dies sind die standardisierten Technologien zur Umsetzung von Webseiten. Die einzelnen Systemkomponenten wie z.B. der Homescreen oder die Einstellungen sind eigene Apps. So kann der Homescreen theoretisch auch als Webseite aufgerufen werden. Die bei der Installation mitgelieferten Apps befinden sich alle in Gaia, daher handelt es sich hierbei um ein grösseres Projekt. In dieser Arbeit werden wir uns nur auf Gaia fokussieren.

\section{Problemstellung / Fragen}
Mozilla hat mittlerweile 1000 Angestellte\footnote{New York Times: Personality and Change Inflamed Mozilla Crisis. \url{http://www.nytimes.com/2014/04/05/technology/personality-and-change-inflamed-crisis-at-mozilla.html}. Aufgerufen am 05. April 2015.}. Nicht alle werden für Firefox OS eingesetzt, obwohl Firefox OS und Firefox hohe Priorität geniessen. Die Wartung und Weiterentwicklung eines Browsers und eines mobilen Betriebssystem erfordert viele Resourcen. Trotzdem expandiert Firefox OS auch in 2015 in weitere Märkte. Neue Märkte bedeuten neue Anforderungen an das Betriebssystem, wobei die Wartung und Neuentwicklungen für bestehende Märkte nicht vernachlässigt werden darf. Bedeutet dies auch, dass es in Gaia mehr Anpassungen gibt als Mitte 2013? Oder gibt es weniger, da die Basis dafür seit Firefox OS 1.0 bereits existiert? Kann Mozilla und die dazugehörige Community den Anforderungen gerecht werden und die Anzahl der Anpassungen konstant halten? Da es sich bei Firefox OS um ein hoch priorisiertes Produkt von Mozilla handelt, stellt sich auch die Frage, wie viele der Anpassungen von Freiwilligen getätigt werden.

Der Fokus dieser Arbeit und der Analysen darin beschränken sich auf Gaia. Änderungen an den darunterliegenden Schichten werden aussenvor gelassen. Viele Angestellte und Freiwillige arbeiten an Lokalisierung (Übersetzung und Anpassung an lokale Gegebenheiten), Design, Support und vielen weiteren Themengebieten. Diese Arbeiten werden in dieser Arbeit nicht eingerechnet, die Datenanalyse beschränkt sich vollständig auf Änderungen im Code, d.h. bei der Programmierung.

\section{Ziele}
Das Ziel dieser Arbeit ist es, obenstehende Fragen anhand von Statistik und Wahrscheinlichkeit zu beantworten. Die Daten werden anhand der gelernten Methoden aus dem Modul Wahrscheinlichkeit und Statistik analysiert und daraus ein Fazit gezogen.