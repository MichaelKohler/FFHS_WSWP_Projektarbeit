\chapter{Daten}

\section{Datenquelle}
Der Source-Code von Gaia ist öffentlich auf der Webseite GitHub.com verfügbar\footnote{GitHub.com: Gaia Repository. \url{https://github.com/mozilla-b2g/gaia}. Aufgerufen am 05. April 2015.}. Git ist ein Versionsverwaltungssystem, welches in vielen Projekten eingesetzt wird. Darin sind alle Änderungen seit Projektstart verzeichnet. Für jede Änderung wird eine neue Version der Datei erstellt, dies nennt sich im Fachjargon \"Commit\". Insgesamt befinden sich auf dem Projekt 45'591 solche Änderungen (Stand: 05. April 2015, 21:25 Uhr). Zu allen Änderungen werden jeweils auch der Author und andere wichtige Informationen gespeichert.

\section{Datenstand}
Die vorliegende Arbeit beschränkt sich auf die Daten zwischen dem 1. Juli 2013 und dem 31. März 2015. Dieser Zeitraum umfasst 21 Monate mit insgesamt 26'097 Code-Änderungen. Dabei wurden insgesamt 52'233'826 Codezeilen hinzugefügt, 46'933'408 wurden gelöscht. Dies bedeutet, dass das gesamte Projekt in dieser Zeit um 5'300'418 Codezeilen gewachsen ist.

\section{Methode der Datensammlung}
Mit dem Werkzeug GitStats\footnote{Sourceforge: GitStats. \url{http://gitstats.sourceforge.net}. Aufgerufen am 05. April 2015.} kann der Versionsverlauf automatisiert analysiert werden. Dabei werden diverse Tabellen mit Daten erstellt. Zudem werden hilfreiche Grafiken erstellt. Der nachfolgende Befehl wurde am 05. April um 15:30 Uhr ausgeführt. Das Datenset wurde nach diesem Zeitpunkt nicht mehr verändert.

\noindent
GitStats kann wie folgt gestartet werden:

\begin{lstlisting}
./gitstats <PfadZumRepository> <Output-Pfad>
\end{lstlisting}

\noindent
Das zweite Argument muss ein lokaler Ordner mit dem Git-Repository sein. Der Output-Pfad ist die Ablage für die generierten Daten.

\begin{table}%
\begin{tabularx}{\textwidth}{p{0.25\textwidth}|r|X|r}
Monat & Commits & Zeilen hinzugefügt & Zeilen gelöscht\\
\hline
2015-03	& 1015	& 65682	& 63254\\
\hline
2015-02	& 1054	& 151113	& 68966\\
\hline
2015-01	& 1051	& 83204	& 38915\\
\hline
2014-12	& 717	& 36177	& 25820\\
\hline
2014-11	& 1137	& 115641	& 68708\\
\hline
2014-10	& 1399	& 698579	& 162472\\
\hline
2014-09	& 1501	& 231320	& 85979\\
\hline
2014-08	& 1649	& 564098	& 163910\\
\hline
2014-07	& 1571	& 97336	& 44629\\
\hline
2014-06	& 1953	& 95207	& 52199\\
\hline
2014-05	& 1397	& 117676	& 74816\\
\hline
2014-04	& 1356	& 248112	& 211322\\
\hline
2014-03	& 1373	& 139912	& 97712\\
\hline
2014-02	& 1231	& 11888136	& 11864080\\
\hline
2014-01	& 1280	& 17914322	& 17860654\\
\hline
2013-12	& 1039	& 4142439	& 2708954\\
\hline
2013-11	& 1344	& 315643	& 187441\\
\hline
2013-10	& 1103	& 12811632	& 12789663\\
\hline
2013-09	& 1114	& 119788	& 66575\\
\hline
2013-08	& 1061	& 2170748	& 267816\\
\hline
2013-07	& 752	& 227061	& 29523\\
\end{tabularx}
\caption{Anzahl Commits pro Monat}
\end{table}%

Maschinell auswertbare Versionen dieser Daten befinden sich im Appendix oder unter \url{http://www.michaelkohler.info/downloads/Projektarbeit_Gaia.zip}.

\section{Überprüfbarkeit}
Die Daten basieren auf öffentlich zugänglichen Datenquellen. Daher kann jeder Interessierte die oben genannten Schritte durchführen und erhält die selben Daten.